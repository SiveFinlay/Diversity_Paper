% Preamble
\documentclass[12pt,a4paper]{article}
\usepackage{enumerate} 	% put in numbers or bullet points
\usepackage{setspace}	% line spacing					
\usepackage{authblk}	% For author affiliations
\usepackage{graphicx} 	% For adding pictures

\usepackage[nomarkers]{endfloat} %Figures and tables at the end of the document
\usepackage{pdflscape}	% for landscape pages
\usepackage{mathtools}	% For equations etc.
\usepackage[osf]{mathpazo} % palatino font package
\usepackage{fixltx2e}	% includes subscripts
\usepackage{ms}     	% load the manuscript style template - may need to change this for PeerJ
\usepackage{hyperref}

\setcounter{secnumdepth}{0} % removes numbers from section headings
\raggedright 			% justify the text on the left only
\pagenumbering{arabic}	% Page numbers

\usepackage[round]{natbib} % author-year citations in round brackets

%-----------------------------------------
%Title page? NB PeerJ say that they don't need a title page
%----------------------------------------
\title{} 

\author{Sive Finlay$^{1,2}$ and Natalie Cooper$^{1,2,*}$}
\affiliation{\noindent{\footnotesize
$^1$ School of Natural Sciences, Trinity College Dublin, Dublin 2, Ireland.\\ 
$^2$ Trinity Centre for Biodiversity Research, Trinity College Dublin, Dublin 2, Ireland.\\
$^*$Corresponding author: ncooper@tcd.ie; Zoology Building, Trinity College Dublin, Dublin 2, Ireland.\\ Tel: +353 1 896 1926.\\}}
\date{}	% To give blank date

\runninghead{Cranial morphological diversity in tenrecs } 

\keywords{geometric morphometrics, golden moles, morphological diversity, tenrecs}
%---------------------------------------------------------------
% Start of document
\begin{document}
\modulolinenumbers[1] 	% Line numbering on every line

\mstitlepage			% Instead of \maketitle you can use the nice template to get it looking like a manuscript
\parindent=1.5em		% Changes paragraph indenting so it's not so big
\addtolength{\parskip}{.3em} % Changes spacing between sections so it's smaller
%****************************************************

%Abstract in a separate document for PeerJ
%****************************************************
\section{Introduction}

	Analysing patterns of morphological diversity has important implications for our understanding of ecological and evolutionary traits. For example, from a functional ecology perspective, morphological characteristics of limbs inform us about locomotory style \citep[e.g.][]{Bou1987} and the trophic niches associated with particular dental morphologies affect speciation and diversification rates through time \citep{Price2012}. Morphological diversity is also an important aspect of evolutionary patterns such as adaptive radiations and convergent evolution. High morphological diversity is a unifying \citep{Losos2010a, Olson2009}, although not defining \citep{Glor2010, Olson2009}, characteristic of adaptive radiations. Furthermore, analysing morphological convergences in groups such as freshwater cichlid fish \citep{Muschick2012} and anole lizards \citep{Mahler2013} gives interesting insights into the relative repeatability of evolution \citep{Losos2011}.

	Although studies of morphological diversity have clear implications for our understanding of ecological and evolutionary patterns, apart from a few examples \citep[e.g.][]{Ruta2013, Goswami2011, Brusatte2008}, it is still common to study morphological diversity from a qualitative rather than quantitative perspective. However, we need to quantify the morphological similarities and differences among species to gain a better understanding of their ecological interactions and evolutionary history. Unfortunately, morphological diversity is difficult to quantify. Studies are inevitably constrained to measure the diversity of specific traits rather than overall morphologies \citep{Roy1997}. In addition, our perception of morphological diversity is influenced by the trait being used. One study of pterosaurs demonstrated that comparing the diversity of different morphological traits using varying methods produced similar results \citep{Foth2012}. However, it remains unclear whether this finding can be applied to all vertebrate groups: in some species, comparing the relative diversity of cranial and limb morphologies may yield different results \citep{Foth2012}. Furthermore, linear measurements of morphological traits can restrict our understanding of overall morphological variation. A distance matrix of measurements between specific points is unlikely to give a completely accurate representation of a three dimensional structure \citep{Rohlf1993}.
	
	These are important limitations to consider but geometric morphometric approaches help to overcome some of the issues associated with traditional morphological studies \citep{Adams2004}. Morphometric studies based on caliper measurements of particular features can only describe a limited set of distances, ratios and angles which often fail to capture the overall shape of a specific structure \citep{Slice2007}. Geometric morphometrics circumvents these issues by using a system of Cartesian landmark coordinates to define anatomical points. This method captures more of the true, overall anatomical shape of particular structures \citep{Mitteroecker2009}. These more detailed approaches are useful tools for studying patterns of morphological diversity.
	
	Here we apply geometric morphometric techniques to quantify morphological diversity in a Family of small mammals, the tenrecs. Tenrecs (Afrosoricida, Tenrecidae) are a morphologically diverse group that is commonly cited as an example of both convergent evolution and an adaptive radiation \citep{Soarimalala2011, Eisenberg1969}. The Family is comprised of 34 species, 31 of which are endemic to Madagascar \citep{Olson2013}. Body masses of tenrecs span three orders of magnitude (2.5 to > 2,000g); a greater range than all other Families, and most Orders, of living mammals \citep{Olson2003}. Within this vast size range there are tenrecs which convergently resemble shrews (\textit{Microgale} tenrecs), moles (\textit{Oryzorictes} tenrecs) and hedgehogs \citep[\textit{Echinops} and \textit{Setifer} tenrecs,][]{Eisenberg1969}. Their similarities include examples of morphological, behavioural and ecological convergence \citep{Soarimalala2011}. Tenrecs are one of only four endemic mammalian clades in Madagascar and the small mammal species they resemble are absent from the island \citep{Garbutt1999}. Therefore, it appears that tenrecs represent an adaptive radiation of species which filled otherwise vacant ecological niches \citep{Soarimalala2011}.
	The similarities among tenrecs and other small mammals are even more remarkable when you consider their phylogenetic history. Tenrecs were originally classified within the general "Insectivora" clade and only molecular studies revealed their true phylogenetic affinities within the Afrotherian mammals \citep{Stanhope1998}. Therefore, despite initial appearances, tenrecs are more closely related to elephants, manatees and aardvarks than they are to shrews, moles or hedgehogs. 

	Although tenrecs are often cited as an example of both an adaptive radiation and exceptional convergent evolution, these claims have not been investigated quantitatively. There are qualitative similarities among the hind limb morphologies of tenrecs and several other unrelated species with similar locomotory styles \citep{Salton2009} but the degree of morphological similarity has not been established. Morphological diversity is an important feature of adaptive radiations \citep{Losos2010a} and it also informs our understanding of convergent phenotypes \citep{Muschick2012}. Therefore, it is important to quantify patterns of morphological diversity in tenrecs to gain an insight into their evolution. My thesis is the first study to address this issue. 
	
	
	We present the first quantitative study of patterns of morphological diversity in tenrecs. We use geometric morphometric techniques \citep{Rohlf1993} to compare cranial morphological diversity in tenrecs to that of their closest relatives, the golden moles (Afrosoricida, Chrysochloridae). 
	We expect tenrecs to be more morphologically diverse than golden moles because tenrecs occupy a wider variety of ecological niches. The tenrec Family includes terrestrial, semi-fossorial, semi-aquatic and semi-arboreal species \citep{Soarimalala2011}. In contrast, all golden moles occupy very similar, fossorial ecological niches \citep{Bronner1995}. Greater ecological variety is often (though not always) correlated with higher morphological diversity \citep{Losos2010a}. 
	
	%Finish off this introduction
%************************************
\section{Materials and Methods}

%Overview section: data, geometric morphometrics, measuring diversity

	The methods we used involved several steps of data collection, geometric morphometrics analyses and comparisons of morphological diversity. For clarity,  Figure \ref{fig:flow} summarises all of these steps and we describe them in detail below.  	
%Methods flow chart
	\begin{figure}[!htbp]
	\centering
	\includegraphics[width=1\linewidth,height=0.8\textheight]{figures/methods_flowchart_paper.png}
		
	\caption[Flowchart diagram of data collection and analysis]
		{Summary of the main steps in our data collection, processing and analysis protocol. Note that the analyses were repeated separately for each set of photographs: skulls in dorsal, ventral and lateral views. The dashed arrows refer to the stage at which we selected a subsample of the tenrecs (including just five species of the \textit{Microgale} Genus) so that we could compare the morphological diversity of this reduced subsample of tenrec species to the diversity of golden moles.}
		
		\label{fig:flow}
		\end{figure}

\subsection{Data collection}
	One of us (SF) used the collections of five museums: the Natural History Museum, London (BMNH), the Smithsonian Institute Natural History Museum, Washington D.C. (SI), the American Museum of Natural History, New York (AMNH), the Museum of Comparative Zoology, Cambridge M.A. (MCZ) and the Field Museum of Natural History, Chicago (FMNH). We recorded species names as they were written on museum specimen labels and then corrected them to match the taxonomy in Wilson and Reeder's Mammal Species of the World \citeyearpar{Wilson2005}. For recently identified species,  which are not included in Wilson and Reeder \citeyearpar{Wilson2005}, we used the taxonomy recorded on the specimen labels. Wilson and Reeder \citeyearpar{Wilson2005} record 30 species of tenrec but more recent studies indicate that there are now 34 species \citep[][]{Olson2013}. The additional species belong to the shrew tenrec (\textit{Microgale}) Genus and represent either recognition of cryptic species boundaries \citep{Olson2004} or discovery of new species \citep{Goodman2006, Olson2009}. Only one of these four recent additions, \textit{M. jobihely}, was present in the museum collections and therefore we could not include the three other newly recognised species in the analyses. 
 	We photographed all of the tenrec and golden mole skulls available in the collections. This included 31 of the 34 species in the tenrec Family and 12 of the 21 species of golden moles \citep{Wilson2005}.
	%Left out the points about male/female and adult/juvenile
	
	We took pictures of the skulls using photographic copy stands consisting of a camera attachment with an adjustable height bar, a flat stage on which to place the specimen and an adjustable light source. To take possible light variability into account, on each day we took a photograph of a white sheet of paper and used the custom white balance function on the camera to set the image as the baseline "white" measurement for those particular light conditions.
		
	We photographed the specimens with a Canon EOS 650D camera fitted with a EF 100 mm f/2.8 Macro USM lens. We used a remote control (H\"ahnel Combi TF) to take the photos to avoid shaking the camera and distorting the images. We photographed the specimens on a black material background with a light source in the top left-hand corner of the photograph. We used small bean bags as necessary to hold the specimens in position while being photographed to ensure that they lay in a flat plane relative to the camera and did not tilt in any direction. We used the grid-line function on the live-view display screen of the camera to position the specimens in the centre of each image. 
	
	We photographed the skulls in three views: dorsal (top of the cranium), ventral (underside of the skull with the palate roof facing upwards) and lateral (right side of the skull). When the right sides of the skulls were damaged or incomplete we photographed the left sides and later reflected the images so that they could be compared to pictures of the right sides \citep[e.g.][]{Barrow2008}.
	

	We converted the raw files to binary (grey scale) images and re-saved them as TIFF files \citep{RHOI2013}. 
	Photographs of the specimens from the American Museum of Natural History and the Smithsonian Institute are available on figshare in separate file sets for the dorsal \citep{Finlay2013d}, ventral \citep{Finlay2013v} and lateral \citep{Finlay2013l} skull pictures along with the mandibles \citep{Finlay2013m}. Copyright restrictions from the other museums prevent public sharing of their images however they are available on request.

%-----------------------------------------
\subsection{Geometric morphometric analyses}

\subsubsection{Landmark placement on images}

	We used a combination of landmark and semilandmark analysis approaches to assess the shape variability in skull.  We used the TPS software suite \citep{Rohlf2013} to digitise landmarks and curves on the photos. We set the scale on each image individually to standardise for the different camera heights that I used when photographing my specimens. We created separate data files for each of the three morphometric analyses (skulls in dorsal, ventral and lateral views). One of us (SF) digitised landmarks and semilandmark points on every image individually. Some specimens were too damaged to use in particular views so there were a different total number of images for each analysis: skulls dorsal (356), ventral (346) and lateral (336).
	%Numbers are from the two_family_disparity script after cleaning up the data to remove damages specimens

	When using semilandmark approaches there is a potential problem of over - sampling: simpler structures will require fewer semilandmarks to accurately represent their shape  \citep{MacLeod2012}. To ensure that we applied a uniform standard of shape representation to each outline segment (i.e. that simple structures would not be over-represented and more complex features would not be under-represented), we followed the method outlined by MacLeod \citeyearpar{MacLeod2012}. For each data set we chose a random selection of photos of specimens which represented the breadth of the morphological data (i.e. specimens from each sub-group of species). We drew the appropriate curves on each specimen and over-sampled the number of points on the curves. We measured the length of the line and regarded that as the 100\%, true length of that outline. We then re-sampled the curves with decreasing numbers of points and measured the length of the outlines. We calculated the length of each re-sampled curve as a percentage of the total length of the curve and then found the average percentage length for that reduced number of semilandmark points across all of the specimens in my test file. We continued this process until I found the minimum number of points that gave a curve length which was at least 95\% accurate.  We repeated these curve-sampling tests for each analysis to determine the minimum number of semilandmark points which would give accurate representations of morphological shape.
	
	Figure (REF) depicts that landmarks and curves which we used for each of the sets of photographs. For landmarks which are defined by dental structures, we used published dental sources \citep{Repenning1967, Eisenberg1969, Nowak1983, MacPhee1987, KnoxJones1992, Davis1997, Querouil2001, Nagorsen2002, Wilson2005, Goodman2006, Karatas2007, Hoffmann2008, Asher2008,  Muldoon2009, Lin2010} where available to identify the number and type of teeth in each species. Detailed descriptions of the landmarks can be found in the supplementary material.
	%Put the landmark tables into supplementary
	
	%Add the landmarks picture: three skulls together
	
	%Maybe put morphometrics error checking into the supplementary as well?
%*********************************
\section{Results and Discussion}

%**********************************
\section{Conclusions}

%*******************************
\section{Acknowledgements}

%********************************
\bibliographystyle{sysbio} 
\bibliography{refs_paper} 

\end{document}