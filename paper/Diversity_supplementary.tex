\documentclass[12pt,a4paper]{article}
\usepackage{enumerate} 	% put in numbers or bullet points
\usepackage{setspace}	% line spacing					
\usepackage{authblk}	% For author affiliations
\usepackage{graphicx} 	% For adding pictures
\usepackage{pdflscape}	% for landscape pages
\usepackage{mathtools}	% For equations etc.
\usepackage[osf]{mathpazo} 
\usepackage{float}		
\floatstyle{plaintop} 	% Force table captions to go above the table
\usepackage{longtable}
\usepackage[margin ={2cm, 2cm, 2cm, 2cm}]{geometry}
\usepackage[round]{natbib}
%\setcounter{secnumdepth}{0} % removes numbers from section headings
\raggedright 			% justify the text on the left only
\pagenumbering{arabic}
\linespread{1.6}
	

\begin{document}

\title{
       Morphological diversity in tenrecs (Afrosoricida, Tenrecidae): Comparing tenrec skull diversity to their closest relatives.\\
       \bigskip
       Supplementary Material }
\author{Sive Finlay and Natalie Cooper}
\date{}
\maketitle
%I could take more out of the main paper and add it to here
%---------------------------------------------------------
%Landmarks 
%---------------------------------------------------------
\section{Landmark descriptions}
	The information below provides further details about the landmarks we used to summarise morphological shape in tenrec and golden mole skulls. Landmark numbers and curve descriptions refer to Figure 2 in the main paper. One of us (SF) placed all of the landmarks on each picture.

\subsection{Skulls: dorsal view}
	Most of our landmarks in this view are relative (type 3) points which represent overall morphological shape but not necessarily homologous biological features \citep{Zelditch2012}. We placed ten landmarks and drew four semilandmark curves to represent the shape of both the braincase (posterior) and nasal (anterior) area of the skulls. Table \ref{tab:skdors} describes how we defined the landmarks and outline curves for our images of skulls in dorsal view.

%Table with skdors landmark descriptions
\begin{table}[h]
	\caption[Skulls: dorsal landmarks]
		{Descriptions of the landmarks (points) and curves (semilandmarks) for the skulls in dorsal view} 
	%Skdors landmarks


\begin{tabular}[t]{p{0.2\textwidth} p{0.75\textwidth}}		
\hline
\textbf{Landmark} & \textbf{Description} \\
\hline
%------------------------------------------------------------
1 + 2 & Left (1) and right (2) anterior points of the premaxilla \\
%------------------------------------------------------------
3 & Anterior of the nasal bones in the midline \\
%------------------------------------------------------------
4 + 5 &	Maximum width of the palate (maxillary) on the left (4) and right (5)\\
%------------------------------------------------------------
6 & Midline intersection between nasal and frontal bones \\
%------------------------------------------------------------
7 + 8 & Widest point of the skull on the left (7) and right (8) \\
%------------------------------------------------------------
9 &	Posterior of the skull in the midline \\
	%Panchetti 2008 and Macholan2008 have different definitions for this one so I need to choose one
%------------------------------------------------------------
10 & Posterior intersection between saggital and parietal sutures \\
%--------------------------------------
\hline
Curve A (12 points) & Outline of the braincase on the left side, between landmarks 7 and 9 (does not include visible features from the lower (ventral) side of the skull) \\

Curve B (10 points) & Outline of the palate on the left side, between landmarks 1 and 4 (outline of the rostrum only, not the shape of the teeth)\\

Curve C (12 points) &	Outline of the braincase on the right side, between landmarks 8 and 9 (does not include visible features from the lower (ventral) side of the skull) \\

Curve D (10 points) & Outline of the palate on the right side, between landmarks 2 and 5 (outline of the rostrum only, not the shape of the teeth)\\
%------------------------------------------------------------
\hline
\end{tabular}
	\label{tab:skdors}
\end{table}
%--------------------------------------------------------------

\subsection{Skulls: ventral view}
	Most of the landmarks in this view are concentrated around the dentition and palate of the animals. We identified 13 landmarks and drew one outline curve (resampled to 60 semilandmark points) around the back of the skull between landmarks 12 and 13. The high variability of our species' basi-cranial regions and difficulties associated with identifying developmentally or functionally homologous points precluded designation of additional landmarks towards the back of the skulls. Table \ref{tab:skvent} outlines the descriptions of the landmarks we used for the ventral skull photos.


% Skulls ventral landmarks
\begin{table}[!htb] 
\caption[Skulls: ventral landmarks]
		{Descriptions of the landmarks (points) and curves (semilandmarks) for the skulls in ventral view.} 
%SkVent landmarks
\begin{tabular}[t]{p{0.2\textwidth} p{0.75\textwidth}}		
\hline
\textbf{Landmark} & \textbf{Description} \\
\hline
%--------------------------------------
1 & Anterior point of the palate\\
%--------------------------------------
2 + 3 & Posterior, lateral extremity of the right (2) and left (3) incisor\\
%--------------------------------------
4 + 5 & Anterior, outer point of the first molar on the right (4) and left (5)\\
%--------------------------------------
6 + 7 & Posterior, outermost point of the last molar surface on the right (6) and left (7) \\
%--------------------------------------
8 & Widest point of the curve of the palatine on the right side\\
%--------------------------------------
9 & Posterior point of the palatine in the midline\\
%--------------------------------------
10 & Widest point of the curve of the palatine on the left side\\
%--------------------------------------
11 & Anterior of the occipital foramen in the midline\\
%--------------------------------------
12 + 13 & Widest (extreme lateral) point of the braincase on the right (12) and left (13)\\
%--------------------------------------
Curve*  & Outline of the back of the skull (between landmarks 12 and 13), 60 points \\
%------------------------------------------------------------
\hline
\end{tabular}
\label{tab:skvent}
\\
*This curve does not necessarily trace homologous features because of the variation in the position of the foramen magnum.
\end{table}
	
%------------------------------------------------------
\newpage
\subsection{Skulls: lateral view}
	We placed nine landmarks on the photographs of skulls in lateral view and also drew two semilandmark curves to represent the shape of the back of the skull (between landmarks 7 and 8, re-sampled to 20 semilandmark points) and the top of the skull (between landmarks 8 and 1, re-sampled to 15 semilandmark points). Table \ref{tab:sklat} describes our definitions for each of the landmark points.
	If specimens were damaged on their right side we reflected photographs of the left lateral side of the skull so that all the images were in the same orientation.

% Skulls lateral landmarks
\begin{table}[!htb]
\caption[Skulls: lateral landmarks]
		{Descriptions of the landmarks (points) and curves (semilandmarks) for the skulls in lateral view.} 
%Sklat landmarks

\begin{tabular}[t]{p{0.2\textwidth} p{0.75\textwidth}}		
\hline
\textbf{Landmark} & \textbf{Description} \\
\hline
%--------------------------------------
1 & Anterior, upper tip of the nasal bone\\
%--------------------------------------
2 & Anterior of the alveolus of the first incisor\\
%--------------------------------------
3 & Lowest point of the first incisor\\
%--------------------------------------
4& Posterior of the alveolus of the last incisor \\
%--------------------------------------
5 & Anterior tip of the alveolus of the first molar\\
%--------------------------------------
6 & Posterior tip of the alveolus of the last molar\\
%--------------------------------------
7 & Lowest point of the basi-occipital (base of the back of the skull)\\
%--------------------------------------
8 & Highest point of the braincase\\
%--------------------------------------
9 & Highest point of the infraorbital foramen\\
%--------------------------------------
\hline
Curve A & Between points 7 and 8  \\
(20 points)& Back of the skull from the lowest to highest points\\
%------------------------------------------------------------
Curve B & Between points 8 and 1  \\
(15 points)&From the highest point of the braincase to the front of the nasal \\
%------------------------------------------------------------
\hline
\end{tabular}
\label{tab:sklat}
\end{table}
%-------------------------------------------
\bibliographystyle{sysbio} 
\bibliography{refs_paper} 
\end{document}