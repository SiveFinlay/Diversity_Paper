\documentclass[12pt,a4paper]{article}
\usepackage{enumerate} 	% put in numbers or bullet points
\usepackage{setspace}	% line spacing					
\usepackage{authblk}	% For author affiliations
\usepackage{graphicx} 	% For adding pictures
\usepackage{pdflscape}	% for landscape pages
\usepackage{mathtools}	% For equations etc.
\usepackage[osf]{mathpazo} 
\usepackage{float}		
\floatstyle{plaintop} 	% Force table captions to go above the table
\usepackage{longtable}
\usepackage[margin ={2cm, 2cm, 2cm, 2cm}]{geometry}
\usepackage[round]{natbib}
%\setcounter{secnumdepth}{0} % removes numbers from section headings
\raggedright 			% justify the text on the left only
\pagenumbering{arabic}
\linespread{1.6}
	

\begin{document}

\title{
       Morphological diversity in tenrecs (Afrosoricida, Tenrecidae): Comparing tenrec skull diversity to their closest relatives.\\
       \bigskip
       Supplementary Material }
\author{Sive Finlay and Natalie Cooper}
\date{}
\maketitle
%I could take more out of the main paper and add it to here
%---------------------------------------------------------
%Landmarks 
%---------------------------------------------------------
\section{Landmark descriptions}
	The information below provides further details about the landmarks we used to summarise morphological shape in tenrec and golden mole skulls. Landmark numbers and curve descriptions refer to Figure 2 in the main paper. One of us (SF) placed all of the landmarks on each picture.

\subsection{Skulls: dorsal view}
	Most of our landmarks in this view are relative (type 3) points which represent overall morphological shape but not necessarily homologous biological features \citep{Zelditch2012}. We placed ten landmarks and drew four semilandmark curves to represent the shape of both the braincase (posterior) and nasal (anterior) area of the skulls. Table \ref{tab:skdors} describes how we defined the landmarks and outline curves for our images of skulls in dorsal view.

%Table with skdors landmark descriptions
\begin{table}[h]
	\caption[Skulls: dorsal landmarks]
		{Descriptions of the landmarks (points) and curves (semilandmarks) for the skulls in dorsal view} 
	\input{tables/skdors.landmarks}
	\label{tab:skdors}
\end{table}
%--------------------------------------------------------------

\subsection{Skulls: ventral view}
	Most of the landmarks in this view are concentrated around the dentition and palate of the animals. We identified 13 landmarks and drew one outline curve (resampled to 60 semilandmark points) around the back of the skull between landmarks 12 and 13. The high variability of our species' basi-cranial regions and difficulties associated with identifying developmentally or functionally homologous points precluded designation of additional landmarks towards the back of the skulls. Table \ref{tab:skvent} outlines the descriptions of the landmarks we used for the ventral skull photos.


% Skulls ventral landmarks
\begin{table}[!htb] 
\caption[Skulls: ventral landmarks]
		{Descriptions of the landmarks (points) and curves (semilandmarks) for the skulls in ventral view.} 
\input{tables/skvent.landmarks}
\label{tab:skvent}
\\
*This curve does not necessarily trace homologous features because of the variation in the position of the foramen magnum.
\end{table}
	
%------------------------------------------------------
\newpage
\subsection{Skulls: lateral view}
	We placed nine landmarks on the photographs of skulls in lateral view and also drew two semilandmark curves to represent the shape of the back of the skull (between landmarks 7 and 8, re-sampled to 20 semilandmark points) and the top of the skull (between landmarks 8 and 1, re-sampled to 15 semilandmark points). Table \ref{tab:sklat} describes our definitions for each of the landmark points.
	If specimens were damaged on their right side we reflected photographs of the left lateral side of the skull so that all the images were in the same orientation.

% Skulls lateral landmarks
\begin{table}[!htb]
\caption[Skulls: lateral landmarks]
		{Descriptions of the landmarks (points) and curves (semilandmarks) for the skulls in lateral view.} 
\input{tables/sklat.landmarks}
\label{tab:sklat}
\end{table}
%-------------------------------------------
\bibliographystyle{sysbio} 
\bibliography{refs_paper} 
\end{document}